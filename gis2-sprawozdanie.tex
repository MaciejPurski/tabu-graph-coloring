\documentclass[12pt,a4paper]{article}
\usepackage[utf8]{inputenc}
\usepackage[polish]{babel}
\usepackage{amsmath}
\usepackage{amsfonts}
\usepackage[T1]{fontenc}
\setlength{\parindent}{0.5cm}
\usepackage{url}
\usepackage{hyperref}
\usepackage{breakurl}


\author{Jakub Mazur, Maciej Purski}
\title{Kolorowanie grafów za pomocą metod heurystycznych (SK16) - GIS sprawozdanie 2}





\begin{document}
\maketitle

\section{Temat projektu}

\indent
Projekt ma na celu implementację algorytmu rozwiązującego problem kolorowania grafu.
Przez kolorowanie rozumiemy podział zbioru wierzchołków $V$ na $k$ rozłącznych klas $C_k$ w taki
sposób, że jeśli dowolne dwa wierzchołki $u$ i $v$ należą do jednej klasy $C_k$ , to wierzchołki te nie
są połączone krawędzią.

Celem algorytmu będzie wyznaczenie takiego podziału na klasy, by liczba $k$ była jak
najmniejsza oraz, by nie występowały konflikty tj. nie można było znaleźć dwóch wierzchołków
połączonych krawędzią, które byłyby w jednej klasie.
Do rozwiązania problemu zostanie wykorzystana metoda heurystyczna \textbf{przeszukiwania z tabu}.

\section{Algorytm}

\textbf{Przeszukiwanie z tabu} jest heurystyczną metodą służącą do rozwiązywanie problemów optymalizacyjnych. Opiera się na przeszukiwaniu lokalnego sąsiedztwa danego rozwiązania. Tego typu metody silną tendencję do zatrzymywania się w optimum lokalnym. Algorytm przeszukiwania z tabu zakłada przechowywanie pewnego zbioru ostatnio odwiedzonych rozwiązań, tak aby przeszukiwanie nie \textit{utknęło} w optimum lokalnym.

Algorytm może być używany do różnego typu problemów optymalizacyjnych. Jego postać ogólna została przedstawiona w \cite{alhe-tabu}. Dla zastosowania algorytmu do rozwiązania problemu kolorowania grafu, należy uszczegółowić jego poszczególne elementy, co jest tematem kolejnych podrozdziałów.

\subsection{Reprezentacja rozwiązania}
\textit{Rozwiązaniem} będziemy nazywali podział zbioru wierzchołków \textit{V} na \textit{k} niepustych podzbiorów, które można interpretować jako \textit{klasy} lub \textit{kolory}.

\subsection{Warunki początkowe}
Początkowa liczba kolorów $k$ będzie stanowiła parametr algorytmu. Wierzchołki zostaną pokolorowane w sposób losowy z gwarancją, że każdy z $k$ kolorów zostanie użyty dokładnie raz. Takie początkowe kolorowanie nie musi być wcale kolorowaniem poprawnym.


\subsection{Sąsiedztwo}
Zgodnie z \cite{coloring} \textbf{sąsiednim} rozwiązaniem $y\in N(x)$ do rozwiązania $x$, nazywamy takie rozwiązanie \textit{y}, które można uzyskać poprzez zmianę przynależności jednego z wierzchołków rozwiązania \textit{x}. Generacja jednego sąsiada polega na wylosowaniu jednej z niepustych klas $C_i$, następnie wylosowaniu wierzchołka $v\in C_i$ oraz liczby $j\in\langle1,k+1\rangle$, gdzie $k$ to liczba aktualnie użytych kolorów (klas). Następnie wierzchołek $v$ jest kolorowany na kolor $j$. Jeżeli wylosowano tę samą klasę, do której należy wierzchołek, losowanie jest powtarzane.

W każdym kroku generowane jest $g$ sąsiadów \textit{punktu roboczego}\footnote{tj. aktualnie przetwarzanego}, gdzie $g$ stanowi parametr algorytmu. Spośród zbioru sąsiadów wybierany jest najlepszy element względem funkcji kosztu, który nie należy do listy tabu. Będzie on nowym punktem roboczym.

\subsection{Lista tabu}
Lista tabu przechowuje rozwiązania już odwiedzone, które były wcześniej punktem roboczym. Mieści ona $s$ elementów, gdzie $s$ stanowi parametr algorytmu.
Jeżeli lista się zapełni, należy usunąć jeden z elementów. Metoda wyboru elementu, który ma zostać usunięty z listy będzie parametrem algorytmu. Dopuszczalne są metody:
\begin{itemize}

\item \textbf{Dostęp losowy} \\
Usuwany jest losowy element z kolejki.
\item \textbf{Kolejka FIFO} \\
Usuwany jest element, który był w kolejce przez największą liczbę iteracji.
\item \textbf{Kolejka priorytetowa} \\
Usuwany jest element \textbf{najlepszy} względem funkcji kosztu.

\end{itemize}

\subsection{Funkcja kosztu}
Aby możliwe było porównywanie poszczególnych rozwiązań, należy zdefiniować funkcję kosztu, która ma być zminimalizowana przez algorytm przeszukiwania z tabu. Pochodzi ona z artykułu \cite{coloring}. Ma ona postać:

\begin{equation}
cost(x) = - \sum_{i=1}^{k}|C_i|^2 + \sum_{i=1}^{k} 2*|C_i|*|E_i|
\end{equation}

gdzie $C_i$ oznacza $i$-tą klasę kolorowania, natomiast $E_i$ oznacza zbiór konfliktujących krawędzi w tej klasie. Pierwszy człon tej funkcji (pierwsza suma) odpowiada za minimalizację liczby klas poprzez \textit{premiowanie} jak najbardziej licznych klas. Drugi człon zas odpowiada za minimalizację liczby konfliktujących krawędzi.


Jedną z głównych obserwacji odnośnie tej funkcji, przedstawionych we wspomnianym artykule, jest to, że jej
minima lokalne odpowiadają podziałom bez konfliktów. Należy zauważyć, że liczba $k$ zawarta we
wzorze nie zakłada z góry ilości użytych kolorów. Minimalizacja liczby użytych kolorów będzie niejako
„efektem ubocznym” minimalizacji wartości funkcji celu.


\subsection{Kryterium stopu}
Algorytm nie ma naturalnego kryterium stopu. Jedną z możliwości, na którą zdecydowaliśmy się w naszej implementacji jest zatrzymanie algorytmu po zadanej liczbie iteracji $t$, która jest parametrem agorytmu.

\subsection{Parametry}
Reasumując, algorytm będzie przyjmował następujące parametry:
\begin{itemize}
\item Początkowa liczba kolorów
\item Liczba iteracji
\item Rozmiar sąsiedztwa
\item Rozmiar listy tabu
\item Metoda zarządzania listą tabu
\end{itemize}

\section{Ewaluacja rozwiązania}

\section{Implementacja}
Algorytm zostanie zaimplementowany w języku C++. Będzie on odczytywał graf w formie tekstowej ze standardowego wejścia i wypisywał rozwiązanie na standardowe wyjście. Możliwe będzie wypisywanie historii wartości funkcji kosztu aktualnego punktu roboczego w celu zbadania jej zmienności w ciągu kolejnych iteracji algorytmu oraz wypisanie najlepszego znalezionego rozwiązania.

Do automatyzacji procesu ewaluacji rozwiązania posłuży skrypt w języku Python, który będzie powtarzał eksperymenty odpowiednią liczbę razy oraz rysował wykresy obrazujące zmianę wartości funkcji kosztu.

\subsection{Użyte struktury danych}
\subsubsection{Reprezentacja grafu}


\bibliography{sources}
\nocite{*}
\bibliographystyle{plain}

\end{document}